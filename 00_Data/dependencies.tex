% Blindtext verwenden
\usepackage{blindtext}

\usepackage{color}

% Code Snippets einfügen
\usepackage{listings}
\lstset{
  backgroundcolor=\color{white}, 
  frame=single,
  keywordstyle=\color{blue},
  numbers=left,
  language=Python}

% weitere Pakete
% Grafiken aus PNG Dateien einbinden
\usepackage{graphicx}
\graphicspath{ {00_Assets/00_Images/} }

% Deutsche Sonderzeichen benutzen 
\usepackage{ngerman}

% deutsche Silbentrennung
\usepackage[ngerman]{babel}

% Eurozeichen einbinden
\usepackage[right]{eurosym}

% Umlaute unter UTF8 nutzen
\usepackage[utf8]{inputenc}

% Zeichenencoding
\usepackage[T1]{fontenc}
\usepackage{lmodern}
\usepackage{fix-cm}

% keine Silbentrennung
\usepackage[none]{hyphenat} 
\sloppy

% Tabellen
\usepackage{tabularx}

% Mathematische Symbole importieren
\usepackage{amssymb}

% Zitieren
\usepackage{cite}
% Festlegung Art der Zitierung
\bibliographystyle{abbrv}

% Paket für Zeilenabstand
\usepackage[onehalfspacing]{setspace}

% für Stichwortverzeichnis
\usepackage{makeidx}

% Indexerstellung
\makeindex

% Package für Abkürzungsverzeichnis

% Auf jeder Seite Überschrift anzeigen
%\pagestyle{headings}

% Seitenränder einstellen
\usepackage[left=30mm, right=30mm, top=35mm, bottom=35mm]{geometry}
