% Blindtext verwenden
\usepackage{blindtext}

% IF statements verwenden
\usepackage{ifthen}

% Texte einfärben
\usepackage{color}
\definecolor{orange}{rgb}{1,0.5,0}

% Code Snippets einfügen
\usepackage{listings}

% Code Formatting für JavaScript
\lstdefinelanguage{JavaScript}{
  keywords={break, case, catch, continue, debugger, default, delete, do, else, finally, for, function, if, in, instanceof, new, return, switch, this, throw, try, typeof, var, void, while, with},
  keywordstyle=\color{blue}\bfseries,
  ndkeywords={class, export, boolean, throw, implements, import, this},
  ndkeywordstyle=\color{darkgray}\bfseries,
  identifierstyle=\color{black},
  sensitive=false,
  comment=[l]{//},
  morecomment=[s]{/*}{*/},
  commentstyle=\color{purple}\ttfamily,
  stringstyle=\color{orange}\ttfamily,
  morestring=[b]',
  morestring=[b]"
}

\lstset{
  backgroundcolor=\color{white},
  frame=single,
  keywordstyle=\color{blue},
  numbers=left,
  language=bash,
  breaklines=true}

% weitere Pakete
% Grafiken aus PNG Dateien einbinden
\usepackage{graphicx}
\graphicspath{ {00_Assets/00_Images/} }

% deutsche Silbentrennung
\usepackage[ngerman]{babel}

% Eurozeichen einbinden
\usepackage[right]{eurosym}

% Umlaute unter UTF8 nutzen
\usepackage[utf8]{inputenc}

% Zeichenencoding
\usepackage[T1]{fontenc}
\usepackage{lmodern}
\usepackage{fix-cm}

% keine Silbentrennung
\usepackage[none]{hyphenat}
\sloppy

% Tabellen
\usepackage{tabularx}
\usepackage{floatrow}
\floatsetup[table]{capposition=top}

% Mathematische Symbole importieren
\usepackage{amssymb}

% Zitieren
\usepackage{cite}
% Festlegung Art der Zitierung
\bibliographystyle{ieeetr}

% Paket für Zeilenabstand
\usepackage[onehalfspacing]{setspace}

% Paket für multicol Umgebung
\usepackage{multicol}

% für Stichwortverzeichnis
\usepackage{makeidx}

% für Abkürzungsverzeichnis
\usepackage[nohyperlinks, printonlyused, withpage, smaller]{acronym}

% Indexerstellung
\makeindex

% Package für Anhänge
\usepackage[toc,page]{appendix}

% Package für Fußtnoten
\usepackage[hang]{footmisc}
\setlength{\footnotemargin}{1em}
\setlength{\skip\footins}{3em}

% Auf jeder Seite Überschrift anzeigen
%\pagestyle{headings}

% Package für Abkürzungsverzeichnis
% --- Abkürzungsverzeichnis: ----------------------------
\usepackage[german]{nomencl}
% Befehl umbenennen in abk
\let\abk\nomenclature
% Deutsche Überschrift
\renewcommand{\nomname}{Abkürzungsverzeichnis}
% Punkte zw. Abkürzung und Erklärung
\setlength{\nomlabelwidth}{.40\hsize}
\renewcommand{\nomlabel}[1]{#1 \dotfill}
% Zeilenabstände verkleinern
\setlength{\nomitemsep}{-\parsep}
\makenomenclature
%--------------------------------------------------------

% Seitenränder einstellen
\usepackage[left=30mm, right=30mm, top=35mm, bottom=35mm]{geometry}
