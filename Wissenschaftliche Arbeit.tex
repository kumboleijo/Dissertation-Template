\documentclass[12pt, a4paper]{article}

% weitere Pakete
% Grafiken aus PNG Dateien einbinden
\usepackage{graphicx}
\graphicspath{ {00_Assets/00_Images/} }

% Deutsche Sonderzeichen benutzen 
\usepackage{ngerman}

% deutsche Silbentrennung
\usepackage[ngerman]{babel}

% Eurozeichen einbinden
\usepackage[right]{eurosym}

% Umlaute unter UTF8 nutzen
\usepackage[utf8]{inputenc}

% Zeichenencoding
\usepackage[T1]{fontenc}
\usepackage{lmodern}
\usepackage{fix-cm}

% keine Silbentrennung
\usepackage[none]{hyphenat} 
\sloppy

% Tabellen
\usepackage{tabularx}

% Mathematische Symbole importieren
\usepackage{amssymb}

% Zitieren
\usepackage{cite}
% Festlegung Art der Zitierung
\bibliographystyle{abbrv}

% Paket für Zeilenabstand
\usepackage[onehalfspacing]{setspace}

% für Stichwortverzeichnis
\usepackage{makeidx}

% Indexerstellung
\makeindex

% Package für Abkürzungsverzeichnis

% Auf jeder Seite Überschrift anzeigen
%\pagestyle{headings}

% Seitenränder einstellen
\usepackage[left=30mm, right=30mm, top=35mm, bottom=35mm]{geometry}

% Variablen
\newcommand{\firma}{<Company>}
\newcommand{\titel}{<Document Title>}
\newcommand{\autor}{Jonas Schmitt}
\newcommand{\arbeitstyp}{<Document Type>}
\newcommand{\arbeitstypeartikel}{<Article>}
\newcommand{\fach}{<Lecture (optional)>}
\newcommand{\betreuertitel}{<Title>}
\newcommand{\betreuer}{<Name>}
\newcommand{\abgabedatum}{<Date>}

\begin{document}

\begin{titlepage}
	\centering
	{\scshape\LARGE \firma \par}
	\vspace{1cm}
	{\scshape\Large \arbeitstyp \par}
	\vspace{1.5cm}
	{\line(1,0){400} \par}
	{\huge\bfseries \titel \par}
	{\line(1,0){400} \par}
	\vspace{2cm}
	{\Large \fach \par }
	\vspace{1cm}
	{\Large\itshape \autor \par}
	\vfill
	{betreut durch \par}
	{\betreuertitel \ \betreuer \par}
	\vfill
	{\large \abgabedatum \par}
\end{titlepage}

\newpage
\pagenumbering{Roman} 
\setcounter{page}{2}

% Abstract einbinden
\section*{Abstract}
Lorem ipsum dolor sit amet, consetetur sadipscing elitr, sed diam nonumy eirmod tempor invidunt ut labore et dolore magna aliquyam erat, sed diam voluptua. At vero eos et accusam et justo duo dolores et ea rebum. Stet clita kasd gubergren, no sea takimata sanctus est Lorem ipsum dolor sit amet. Lorem ipsum dolor sit amet, consetetur sadipscing elitr, sed diam nonumy eirmod tempor invidunt ut labore et dolore magna aliquyam erat, sed diam voluptua. At vero eos et accusam et justo duo dolores et ea rebum. Stet clita kasd gubergren, no sea takimata sanctus est Lorem ipsum dolor sit amet. Lorem ipsum dolor sit amet, consetetur sadipscing elitr, sed diam nonumy eirmod tempor invidunt ut labore et dolore magna aliquyam erat, sed diam voluptua. At vero eos et accusam et justo duo dolores et ea rebum. Stet clita kasd gubergren, no sea takimata sanctus est Lorem ipsum dolor sit amet.   

\newpage

\tableofcontents

\newpage

\addcontentsline{toc}{section}{Sperrvermerk}
\section*{Sperrvermerk}
\arbeitstypeartikel \ vorliegende \arbeitstyp \ beinhaltet vertrauliche Informationen der \firma. Veröffentlichung und Vervielfältigung ist ohne vorherige schriftliche Genehmigung der \firma \ nicht gestattet. \arbeitstypeartikel \ \arbeitstyp \ ist nur den Gutachtern und den Mitgliedern des Prüfungsausschusses zugänglich zu machen.
\\\\\\\\
\vspace{5cm}
\begin{tabularx}{\textwidth}[b]{p{5cm} X p{5cm}} \cline{1-1} \cline{3-3}
Datum & & \betreuer
\end{tabularx}
\newpage

\addcontentsline{toc}{section}{Eidesstaatliche Erklärung}
\section*{Eidesstaatliche Erklärung}
Hiermit versichere ich, dass ich die vorliegende Arbeit selbstständig und ohne Benutzung anderer als der angegebenen Hilfsmittel angefertigt habe. Alle Stellen, die wörtlich oder sinngemäß aus anderen Schriften entnommen wurden, sind als solche kenntlich gemacht. 
\\\\
Die Arbeit ist noch nicht veröffentlicht oder anderweitig für Prüfungszwecke vorgelegt worden.
\\\\\\\\
\vspace{5cm}
\begin{tabularx}{\textwidth}[b]{p{5cm} X p{5cm}} \cline{1-1} \cline{3-3}
Datum & & \autor
\end{tabularx}

\newpage
\addcontentsline{toc}{section}{Abbildungsverzeichnis}
\listoffigures

\newpage

\addcontentsline{toc}{section}{Tabellenverzeichnis}
\listoftables

\newpage

\setcounter{page}{1}
\pagenumbering{arabic} 

% Hier werden die einzelnen Kapitel eingelesen
\section{Einführung}
\subsection{Themenstellung}
Lorem ipsum dolor sit amet, consetetur sadipscing elitr, sed diam nonumy eirmod tempor invidunt ut labore et dolore magna aliquyam erat, sed diam voluptua. At vero eos et accusam et justo duo dolores et ea rebum. Stet clita kasd gubergren, no sea takimata sanctus est Lorem ipsum dolor sit amet. Lorem ipsum dolor sit amet, consetetur sadipscing elitr, sed diam nonumy eirmod tempor invidunt ut labore et dolore magna aliquyam erat, sed diam voluptua. At vero eos et accusam et justo duo dolores et ea rebum. Stet clita kasd gubergren, no sea takimata sanctus est Lorem ipsum dolor sit amet. Lorem ipsum dolor sit amet, consetetur sadipscing elitr, sed diam nonumy eirmod tempor invidunt ut labore et dolore magna aliquyam erat, sed diam voluptua. At vero eos et accusam et justo duo dolores et ea rebum. Stet clita kasd gubergren, no sea takimata sanctus est Lorem ipsum dolor sit amet.   
\subsection{Vorgehensweise}
Lorem ipsum dolor sit amet, consetetur sadipscing elitr, sed diam nonumy eirmod tempor invidunt ut labore et dolore magna aliquyam erat, sed diam voluptua. At vero eos et accusam et justo duo dolores et ea rebum. Stet clita kasd gubergren, no sea takimata sanctus est Lorem ipsum dolor sit amet. Lorem ipsum dolor sit amet, consetetur sadipscing elitr, sed diam nonumy eirmod tempor invidunt ut labore et dolore magna aliquyam erat, sed diam voluptua. At vero eos et accusam et justo duo dolores et ea rebum. Stet clita kasd gubergren, no sea takimata sanctus est Lorem ipsum dolor sit amet. Lorem ipsum dolor sit amet, consetetur sadipscing elitr, sed diam nonumy eirmod tempor invidunt ut labore et dolore magna aliquyam erat, sed diam voluptua. At vero eos et accusam et justo duo dolores et ea rebum. Stet clita kasd gubergren, no sea takimata sanctus est Lorem ipsum dolor sit amet.   
\subsection{Zielsetzung}
Lorem ipsum dolor sit amet, consetetur sadipscing elitr, sed diam nonumy eirmod tempor invidunt ut labore et dolore magna aliquyam erat, sed diam voluptua. At vero eos et accusam et justo duo dolores et ea rebum. Stet clita kasd gubergren, no sea takimata sanctus est Lorem ipsum dolor sit amet. Lorem ipsum dolor sit amet, consetetur sadipscing elitr, sed diam nonumy eirmod tempor invidunt ut labore et dolore magna aliquyam erat, sed diam voluptua. At vero eos et accusam et justo duo dolores et ea rebum. Stet clita kasd gubergren, no sea takimata sanctus est Lorem ipsum dolor sit amet. Lorem ipsum dolor sit amet, consetetur sadipscing elitr, sed diam nonumy eirmod tempor invidunt ut labore et dolore magna aliquyam erat, sed diam voluptua. At vero eos et accusam et justo duo dolores et ea rebum. Stet clita kasd gubergren, no sea takimata sanctus est Lorem ipsum dolor sit amet.   
\subsection{Stand der Technik}
Lorem ipsum dolor sit amet, consetetur sadipscing elitr, sed diam nonumy eirmod tempor invidunt ut labore et dolore magna aliquyam erat, sed diam voluptua. At vero eos et accusam et justo duo dolores et ea rebum. Stet clita kasd gubergren, no sea takimata sanctus est Lorem ipsum dolor sit amet. Lorem ipsum dolor sit amet, consetetur sadipscing elitr, sed diam nonumy eirmod tempor invidunt ut labore et dolore magna aliquyam erat, sed diam voluptua. At vero eos et accusam et justo duo dolores et ea rebum. Stet clita kasd gubergren, no sea takimata sanctus est Lorem ipsum dolor sit amet. Lorem ipsum dolor sit amet, consetetur sadipscing elitr, sed diam nonumy eirmod tempor invidunt ut labore et dolore magna aliquyam erat, sed diam voluptua. At vero eos et accusam et justo duo dolores et ea rebum. Stet clita kasd gubergren, no sea takimata sanctus est Lorem ipsum dolor sit amet.   
\section{Grundlagen}
\subsection{Einführung}
\subsection{Begriffserklärungen}

\newpage
\setcounter{page}{1}
\pagenumbering{roman} 

\addcontentsline{toc}{section}{Literatur}
\bibliography{02_Bib/mybib}

\end{document}
